---
id: tkz-euclide-ejemplo-27
title: "Ortocentro"
description: "Calcula el ortocentro y traza las tres altitudes de un triángulo."
keywords: [ortocentro, altitudes, triángulo]
tags: [tkzDefTriangle,tkzDefLine,tkzDrawPolygon,tkzGetPoint]
sort: 27
---
\documentclass[tikz,border=2mm]{standalone}
\usepackage{tkz-euclide}

\begin{document}
    \begin{tikzpicture}
        % Define la base AB del triángulo.
        \tkzDefPoint(0,0){A}
        \tkzDefPoint(5,0){B}
    
        % Construye un triángulo isósceles fijando dos ángulos iguales (65° y 65°).
        \tkzDefTriangle[two angles = 65 and 65](A,B)
            \tkzGetPoint{C}
    
        % Encuentra el ortocentro O del triángulo ABC.
        \tkzDefTriangleCenter[ortho](A,B,C)
            \tkzGetPoint{O}
    
        % Traza las rectas de altitud desde cada vértice (A, B, C)
        % hacia el lado opuesto, obteniendo puntos a, b y c en dichas rectas.
        \tkzDefLine[altitude](C,A,B)
            \tkzGetPoint{a}
        \tkzDefLine[altitude](A,B,C)
            \tkzGetPoint{b}
        \tkzDefLine[altitude](B,C,A)
            \tkzGetPoint{c}
    
        % Dibuja las altitudes como segmentos punteados.
        \tkzDrawSegment[dashed](A,a)
        \tkzDrawSegment[dashed](B,b)
        \tkzDrawSegment[dashed](C,c)
    
        % Marca los ángulos rectos donde cada altitud es perpendicular al lado.
        \tkzMarkRightAngle(A,a,B)
        \tkzMarkRightAngle(B,b,A)
        \tkzMarkRightAngle(C,c,B)
    
        % Dibuja el ortocentro y el triángulo.
        \tkzDrawPoints[red](O)
        \tkzDrawPoints(A,B,C)
        \tkzDrawPolygon(A,B,C)
    
        % Etiqueta los puntos principales.
        \tkzLabelPoints(A,B)
        \tkzLabelPoints[above](C)
    \end{tikzpicture}
\end{document}
