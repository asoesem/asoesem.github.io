---
id: tkz-euclide-ejemplo-14
title: Bisector
description: "Construye y muestra la bisectriz de un ángulo y su construcción auxiliar."
keywords: [bisectriz, angulo, construccion]
tags: [tkzDefLine,tkzShowLine,bisector]
sort: 14
---
\documentclass[tikz,border=2mm]{standalone}
\usepackage{tkz-euclide}

\begin{document}
    \begin{tikzpicture}
        % Define conjunto de puntos A,B,C.
        \tkzDefPoints{
            2/1/A,
            3/4/B,
            5/0/C%
        }
    
        % Dibuja los puntos A,B,C.
        \tkzDrawPoints(A,B,C)
    
        % Dibuja rectas AB, AC
        % Las rectas se exntienden 0.5cm del extremo final.
        \tkzDrawLines[add=0 and .5](A,B A,C)
    
        % Define la bisectriz Aa del anglulo ∠BAC.
        \tkzDefLine[bisector,normed](B,A,C)
            \tkzGetPoint{a} % Punto en la bisectriz.
    
        % Dibuja la bisectriz Aa, en línea punteada,
        % con flechas en el extremo final, que se extiende 4cm.
        \tkzDrawLine[-Latex,dashed,add=0 and 4](A,a)
    
        % Muestra la construcción geométrica de la bisectriz en rojo.
        \tkzShowLine[bisector,gap=4,size=2,color=red](B,A,C)
    
        % Etiqueta los puntos A,B,C.
        \tkzLabelPoints[left](A,B) % a la izquierda.
        \tkzLabelPoints[below](C)  % por debajo.
    \end{tikzpicture}
\end{document}