---
id: tkz-euclide-ejemplo-29
title: "Incentro"
description: "Calcula el incentro y traza las tres bisectrices, marcando pares de ángulos iguales."
keywords: [incentro, bisectriz, bisector, isosceles]
tags: [tkzDefTriangle, tkzDefTriangleCenter,tkzDefLine,tkzGetPoint]
sort: 29
---
\documentclass[tikz,border=2mm]{standalone}
\usepackage{tkz-base}
\usepackage{tkz-euclide}

\begin{document}
    \begin{tikzpicture}
        % Define la base AB.
        \tkzDefPoint(0,0){A}
        \tkzDefPoint(5,0){B}

        % Construye el triángulo con dos ángulos iguales (65° y 65°).
        \tkzDefTriangle[two angles = 65 and 65](A,B)
            \tkzGetPoint{C}

        % Calcula el incentro O (intersección de las bisectrices).
        \tkzDefTriangleCenter[in](A,B,C)
            \tkzGetPoint{O}

        % Define las tres bisectrices desde A, B y C.
        \tkzDefLine[bisector](C,A,B)
            \tkzGetPoint{a}
        \tkzDefLine[bisector](C,B,A)
            \tkzGetPoint{b}
        \tkzDefLine[bisector](A,C,B)
            \tkzGetPoint{c}

        % Dibuja las bisectrices como segmentos punteados con flecha.
        \tkzDrawSegment[dashed,add=0 and 0.1,-Stealth](A,a)
        \tkzDrawSegment[dashed,add=0 and 0.1,-Stealth](B,b)
        \tkzDrawSegment[dashed,add=0 and 0.1,-Stealth](C,c)

        % Marca ángulos iguales a cada lado de las bisectrices.
        \tkzMarkAngles[mark=|,red](B,A,O O,A,C)
        \tkzMarkAngles[mark=|,red](C,B,O O,B,A)
        \tkzMarkAngles[mark=||,blue](A,C,O O,C,B)

        % Dibuja y etiqueta puntos y triángulo.
        \tkzDrawPoints[red](O)
        \tkzDrawPoints(A,B,C)
        \tkzDrawPolygon(A,B,C)
        \tkzLabelPoints(A,B)
        \tkzLabelPoints[above](C)
    \end{tikzpicture}
\end{document}
