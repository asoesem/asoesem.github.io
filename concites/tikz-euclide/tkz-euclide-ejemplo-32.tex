---
id: tkz-euclide-ejemplo-32
title: "Polígonos II"
description: "Combina un polígono y una polilínea, añadiendo trazos punteados para completar conexiones."
keywords: [poligono, polilinea, linea, punteada]
tags: [tkzDrawPolygon, tkzDrawPolySeg]
sort: 32
---
\documentclass[tikz,border=2mm]{standalone}
\usepackage{tkz-base}
\usepackage{tkz-euclide}

\begin{document}
    \begin{tikzpicture}
        % Define dos conjuntos de puntos para polígonos y polilíneas.
        \tkzDefPoints{
            0/0/A1,
            3/0/B1,
            3/3/C1,
            0/3/D1,
            1/1/A2,
            4/1/B2,
            4/4/C2,
            1/4/D2%
        }
    
        % Dibuja el polígono A1B1C1D1 (cerrado) y la polilínea B2–C2–D2 (abierta).
        \tkzDrawPolygon(A1,B1,C1,D1)
        \tkzDrawPolySeg(B2,C2,D2)
    
        % Completa la polilínea A2–B2–D2 con trazo punteado y conecta A1 con A2.
        \tkzDrawPolySeg[dashed](D2,A2,B2)
        \tkzDrawSegment[dashed](A1,A2)
    
        % Conexiones entre vértices correspondientes de ambos conjuntos.
        \tkzDrawSegment(B1,B2)
        \tkzDrawSegment(C1,C2)
        \tkzDrawSegment(D1,D2)
    \end{tikzpicture}
\end{document}
