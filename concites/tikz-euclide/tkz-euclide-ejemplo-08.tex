---
id: tkz-euclide-ejemplo-08
title: Múltiples Segmentos
description: "Traza varios segmentos, marcando longitudes iguales con rayitas."
keywords: [segmentos, congruencia, marcas, triángulo]
tags: [tkzDrawSegments,tkzMarkSegment,tkzLabelPoints]
sort: 8
---
\documentclass[tikz,border=2mm]{standalone}
\usepackage{tkz-euclide}

\begin{document}
    \begin{tikzpicture}
        % Define el sistema de coordenadas ortogonales.
        \tkzInit[xmin=0,xmax=6,ymin=0,ymax=6]
    
        % Define conjunto de puntos A,B,C.
        \tkzDefPoints{
            1/1/A,
            5/1/B,
            3/4/C%
        }
    
        % Dibuja los puntos A, B, C.
        \tkzDrawPoints(A,B,C)
    
        % Dibjua los segmentos AB, BC, y CA
        \tkzDrawSegments(A,B B,C C,A)
    
        % Marca el segmento AB con una rayita
        \tkzMarkSegment[mark=|](A,B)
    
        % Marca los segmentos AC y BC con dos rayitas
        \tkzMarkSegment[mark=||](A,C)
        \tkzMarkSegment[mark=||](B,C)
    
        % Etiqueta los puntos A, B, y C
        \tkzLabelPoints[left](A)  % izquierda.
        \tkzLabelPoints[right](B) % derecha.
        \tkzLabelPoints[above](C) % arriba
    \end{tikzpicture}
\end{document}