---
id: tkz-euclide-ejemplo-15
title: Altitud
description: "Traza la altitud de un triángulo desde un vértice al lado opuesto."
keywords: [altitud, triangulo, perpendicular]
tags: [tkzDrawSegments,tkzDefLine,tkzGetPoint,altitude]
sort: 15
---
\documentclass[tikz,border=2mm]{standalone}
\usepackage{tkz-base}
\usepackage{tkz-euclide}

\begin{document}
    \begin{tikzpicture}
        % Define los puntos A,B
        \tkzDefPoint(1,1){A}
        \tkzDefPoint(5,1){B}
    
        % Define el punto C relativo a A
        % C está a 4cm de A a 60 grados
        \tkzDefShiftPoint[A](60:4){C}
    
        % Define la altitud CP del triángulo △ACB
        \tkzDefLine[altitude](A,C,B)
            \tkzGetPoint{P} % punto de altitud
    
        % Dibuja los puntos A,B,C,P
        \tkzDrawPoints(A,B,C,P)
    
        % Dibuja segmentos AB, BC, CA
        \tkzDrawSegments(A,B B,C C,A)
    
        % Dibuja el segmento punteado CP
        \tkzDrawSegment[dashed](C,P)
    
        % Etiqueta los puntos A,B,C,P en posiciones convenientes
        \tkzLabelPoints[left](A)  % izquierda.
        \tkzLabelPoints[right](B) % derecha.
        \tkzLabelPoints[above](C) % encima.
        \tkzLabelPoints[below](P) % debajo.
    \end{tikzpicture}
\end{document}