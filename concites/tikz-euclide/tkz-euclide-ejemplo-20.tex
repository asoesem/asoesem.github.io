---
id: tkz-euclide-ejemplo-20
title: "Marcar Ángulos 1"
description: "Marca con arcos iguales dos ángulos del triángulo para indicar congruencia."
keywords: [angulo, congruencia, triangulo, isosceles]
tags: [tkzMarkAngle]
sort: 20
---
\documentclass[tikz,border=2mm]{standalone}
\usepackage{tkz-base}
\usepackage{tkz-euclide}

\begin{document}
    \begin{tikzpicture}
        % Define tres puntos A, B y C que formarán un triángulo.
        \tkzDefPoints{
            1/1/A,
            5/1/B,
            3/4/C%
        }
    
        % Dibuja los puntos del triángulo.
        \tkzDrawPoints(A,B,C)
    
        % Dibuja los lados del triángulo ABC.
        \tkzDrawSegments(A,B B,C C,A)
    
        % Marca con doble trazo dos ángulos iguales: ∠BAC y ∠CBA
        % (esto sugiere que el triángulo es isósceles si se quisiera).
        \tkzMarkAngle[mark=||](B,A,C)  % ∠BAC
        \tkzMarkAngle[mark=||](C,B,A)  % ∠CBA
    
        % Etiqueta los puntos A, B y C en posiciones convenientes.
        \tkzLabelPoints[left](A)   % a la izquierda
        \tkzLabelPoints[right](B)  % a la derecha
        \tkzLabelPoints[above](C)  % por encima
    \end{tikzpicture}
\end{document}
